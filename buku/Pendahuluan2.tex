\chapter{Pendahuluan}
\section{Latar Belakang}
Penyakit kardiovaskular adalah salah satu penyakit yang memiliki angka kematian tertinggi. Berdasarkan data WHO setiap tahunnya terdapat 17,5 juta orang meninggal karena kardiovaskular. Angka ini merupakan 31 persen dari keseluruhan angka kematian di seluruh dunia[]. Hal ini dikarenakan penyakit jantung adalah penyakit yang gejala awalnya seringkali tidak disadari oleh penderitanya dan juga dapat disebabkan karena adanya faktor keturunan. Dalam kasusnya penyakit jantung memiliki berbagai macam bentuk, bisa disebabkan karena terjadinya penyumbatan pembuluh darah jantung atau biasa disebut kardiovaskular, ada juga yang disebabkan karena kelainan jantung yang lain seperti terjadinya denyut jantung yang tidak biasa disebabkan adanya aktivitas elektromagnetik jantung yang tidak biasa. Dari berbagai jenis penyakit jantung ini dibutuhkanlah keahlian khusus untuk mengetahui gejala penyakit jantung yang sedang berlangsung atau yang sedang dialami. Selama ini untuk mengetahui adanya penyakit jantung yang terjadi dalam diri seseorang hanya dapat dilakukan oleh dokter spesialis jantung dan kemudian dibutuhkan analisa lebih lanjut tentang hasil EKG yang diperoleh dari pasien. PVC atau Premature Ventricular Contractions adalah salah satu jenis kelainan jantung yang disebabkan terjadinya kontraksi ventrikel jantung yang tidak wajar. Bagi orang yang tidak memiliki riwayat penyakit jantung, gejala PVC mungkin tidak begitu terasa. Tetapi jika dibiarkan PVC dapat menjadi gejala awal kelainan jantung lain yang lebih serius seperti Ventricle Fibrillation atau Ventricle Tachycardia. Metode untuk mendeteksi PVC sudah sangat banyak, tetapi masih sedikitnya aplikasi atau system yang dapat mendeteksi PVC secara realtime. Sebagaimana yang kita ketahui bahwa penyakit jantung adalah penyakit yang dapat muncul secara tiba-tiba dan tidak semua rasa sakit pada jantung bersumber dari jenis penyakit yang sama. Oleh karena itu deteksi dengan membaca gelombang EKG jantung menjadi satu-satunya jalan yang cukup efektif hingga saat ini. Selain itu pembacaan EKG juga hanya dapat dilakukan oleh dokter spesialis jantung atau memang orang yang mempelajari bagaimana membaca gelombang EKG dan menganalisa jenis penyakit atau kelainan jantung yang terjadi. Keterbatasan-keterbatasan inilah yang kemudian mendorong penulis dalam melakukan penelitian ini. 
\section{Perumusan Masalah}
Berikut rumusan masalah yang ingin saya angkat adalah
\begin{enumerate}
    \item Bagaimana memilih algoritma feature extraction yang tepat untuk sistem yang dibangun secara real time?;
    \item Bagaimana membangun algoritma feature extraction yang memiliki efektifitas waktu yang singkat dengan akurasi yang optimal?;
    \item Bagaimana membangun keseluruhan sistem deteksi dini PVC menggunakan algoritma yang telah dibangun?;
\end{enumerate}
\section{Tujuan}
Berikut adalah tujuan yang ingin dicapai pada penulisan proposal.
\begin{enumerate}
    \item Untuk mengetahui mengapa ini terjadi;
    \item Untuk mempelajari proses kejadian masalah;
    \item Untuk melihat dampak yang dipengaruhi oleh kejadian ini.
\end{enumerate}
\section{Hipotesis (opsional)}
Hipotesis dari tulisan ini adalah
\begin{enumerate}
    \item Masalah timbul karena A;
    \item Hasil numeriknya menuju $x \rightarrow \infty$
\end{enumerate}
\iflogTA
\else
\section{Rencana Kegiatan}
Rencana kegitana yang akan saya lakukan adalah sebagia berikut:
\begin{itemize}
    \item Studi literatur
    \item Memeriksa hasil
\end{itemize}
\section{Jadwal Kegiatan}
The table \ref{table:1} is an example of referenced \LaTeX elements. Laporan proposal ini akan dijadwalkan sesuai dengan tabel yang diberikna berikutnya. 

 
\begin{table}[h!]
  \centering
  \begin{tabular}{|c|m{2.5cm}|m{0.01cm}|m{0.01cm}|m{0.01cm}|m{0.01cm}|m{0.01cm}|m{0.01cm}|m{0.01cm}|m{0.01cm}|m{0.01cm}|m{0.01cm}|m{0.01cm}|m{0.01cm}|m{0.01cm}|m{0.01cm}|m{0.01cm}|m{0.01cm}|m{0.01cm}|m{0.01cm}|m{0.01cm}|m{0.01cm}|m{0.01cm}|m{0.01cm}|m{0.01cm}|m{0.01cm}|}
    \hline
    \multirow{2}{*}{\textbf{No}} & \multirow{2}{*}{\textbf{Kegiatan}} & \multicolumn{24}{|c|}{\textbf{Bulan ke-}} \\
    \hhline{~~------------------------}
    {} & {} & \multicolumn{4}{|c|}{\textbf{1}} & \multicolumn{4}{|c|}{\textbf{2}} & \multicolumn{4}{|c|}{\textbf{3}} & \multicolumn{4}{|c|}{\textbf{4}} & \multicolumn{4}{|c|}{\textbf{5}} & \multicolumn{4}{|c|}{\textbf{6}}\\
    \hline
    1 & Studi Literatur & \cellcolor{blue!25} & \cellcolor{blue!25} & \cellcolor{blue!25} & \cellcolor{blue!25}& \cellcolor{blue!25} & \cellcolor{blue!25} & \cellcolor{blue!25} & \cellcolor{blue!25}& \cellcolor{blue!25} & \cellcolor{blue!25} & \cellcolor{blue!25} & \cellcolor{blue!25}& \cellcolor{blue!25} & \cellcolor{blue!25} & \cellcolor{blue!25} & \cellcolor{blue!25}& \cellcolor{blue!25} & \cellcolor{blue!25} & \cellcolor{blue!25} & \cellcolor{blue!25}& \cellcolor{blue!25} & \cellcolor{blue!25} & \cellcolor{blue!25} & \cellcolor{blue!25}\\
    \hline
    2 & Pengumpulan Data & \cellcolor{blue!25} & \cellcolor{blue!25} & \cellcolor{blue!25} & \cellcolor{blue!25} & {} & {} & {} & {} & {} & {} & {} & {}& {} & {} & {} & {}& {} & {} & {} & {}& {} & {} & {} & {}\\
    \hline
    3 & Analisis dan Perancangan Sistem &  {} & {} & {} & {}  & \cellcolor{blue!25} & \cellcolor{blue!25} & \cellcolor{blue!25} & \cellcolor{blue!25} & \cellcolor{blue!25} & \cellcolor{blue!25} & \cellcolor{blue!25} & \cellcolor{blue!25} & {} & {} & {} & {}& {} & {} & {} & {}& {} & {} & {} & {}\\
    \hline
    4 & Implementasi Sistem &  {} & {} & {} & {} & {} & {} & {} & {}& \cellcolor{blue!25} & \cellcolor{blue!25} & \cellcolor{blue!25} & \cellcolor{blue!25} & \cellcolor{blue!25} & \cellcolor{blue!25} & \cellcolor{blue!25} & \cellcolor{blue!25} & {} & {} & {} & {}& {} & {} & {} & {}\\
    \hline
    5 & Analisa Hasil Implementasi &  {} & {} & {} & {} & {} & {} & {} & {}& {} & {} & {} & {} & \cellcolor{blue!25} & \cellcolor{blue!25} & \cellcolor{blue!25} & \cellcolor{blue!25} & \cellcolor{blue!25} & \cellcolor{blue!25} & \cellcolor{blue!25} & \cellcolor{blue!25} & {} & {} & {} & {}\\
    \hline
    6 & Penulisan Laporan & {} & {} & {} & {} & \cellcolor{blue!25} & \cellcolor{blue!25} & \cellcolor{blue!25} & \cellcolor{blue!25}& \cellcolor{blue!25} & \cellcolor{blue!25} & \cellcolor{blue!25} & \cellcolor{blue!25}& \cellcolor{blue!25} & \cellcolor{blue!25} & \cellcolor{blue!25} & \cellcolor{blue!25}& \cellcolor{blue!25} & \cellcolor{blue!25} & \cellcolor{blue!25} & \cellcolor{blue!25}& \cellcolor{blue!25} & \cellcolor{blue!25} & \cellcolor{blue!25} & \cellcolor{blue!25}\\
    \hline
  \end{tabular}
  \caption{Jadwal kegiatan proposal tugas akhir}
  \label{table:1}
\end{table}

\fi