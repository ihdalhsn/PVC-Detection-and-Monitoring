\chapter*{Abstrak}
%--Overview-- \\
Premature Ventricular Contractions (PVCs) adalah salah satu jenis kelainan jantung yang disebabkan terjadinya kontraksi ventrikel jantung yang tidak biasa, menyebabkan pola denyut jantung menjadi tidak normal atau aritmia dan jika dibiarkan PVC dapat menjadi gejala awal kelainan jantung lain yang lebih berbahaya. Dalam beberapa dekade ini telah banyak diajukan metode untuk melakukan deteksi terjadinya PVC. Pada umumnya metode deteksi ini menganalisa sinyal EKG (Elektrokardiogram) dari pasien. Metode deteksi terdiri dari 4 tahap, yaitu \textit{pre-processing} terhadap sinyal EKG, \textit{feature extraction} \textit{classification}. 
%--Problem-- \\
Nilai akurasi yang diperoleh dari keseluruhan proses deteksi sangat dipengaruhi oleh akurasi bentuk sinyal EKG yang dihasilkan pada tahap \textit{feature extraction}. Oleh karena itu penggunaan algoritma \textit{feature extraction} yang akurat menjadi penting. Dari sekian banyak litaratur yang mengajukan metode deteksi PVC, banyak diantaranya hanya berfokus pada mencari nilai akurasi pada tahap \textit{classification}. Padahal metode \textit{classification} adalah metode yang cukup lama memakan waktu dalam proses. 
%--Objective-- \\
Maka dari itu dalam tugas akhir ini penulis akan melakukan analisis dan pengembangan terhadap algoritma \textit{feature extraction} khusus untuk mendeteksi terjadinya PVC. Dengan adanya algoritma \textit{feature extraction} yang akurat, diharapkan dapat tetap menghasilkan hasil deteksi PVC yang akurat meskipun menggunakan proses \textit{classification} yang sederhana. Sehingga hal ini akan membuat proses deteksi menjadi lebih efisien. 
%--Methodology-- \\
%--Outcome-- \\
Algoritma ekstraksi ciri yang akan digunakan dalam penelitian ini adalah \textit{Windowing Algorithm} yang diusulkan oleh Muhammad Umer(2014) dan algoritma klasifikasi PVC yang diusulkan oleh Ik-Sung Cho(2013). Berdasarkan hasil pengujian dari penerapan algoritma di atas, penelitian ini memperoleh nilai akurasi deteksi sebesar xx\%.
  
\vspace{0.5 cm}
\begin{flushleft}
{\textbf{Kata Kunci:} PVCs, Feature Extraction, Windowing Algorithm.}
\end{flushleft}